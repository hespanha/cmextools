\begin{quote}
  \texttt{[...]=createGateway('parameter name 1',value,'parameter name 2',value,...);}
\end{quote}
Creates a set of gateway cmex functions (which we call \textbf{cmex-functions})
that internally call C functions (which we call the \textbf{c-functions}) that
actually perform the necessary computations.
 
The inputs an outputs to the cmex-functions are specified by a \textbf{template}
that specifies the types and sizes of the input-output matlab arrays.
The sizes may be left as variables, which are determined at run-time,
in which case the "unknown" sizes must also be passed as inputs to the
c-functions.
 
The cmex-functions may (optionally) be encapsulated as a matlab \textbf{class},
which is automatically created. When the c-functions are part of a dynamic library,
this permits the cmex-functions to share variables that are retained across
multiple calls to the cmex-functions.
 
Several options are possible for how the cmex-functions call the c-function
1) The c-functions may be included directly into the mex-functions code;
   and called directly from the gateway function.
   In this case, source code must be provided for each c-function to be
   included in the gateway function using a \#include preprocessor directive.
2) The c-functions are part of a dynamic library that is linked to the
   cmex-functions in run time.
   In this case, source code for the dynamic Library must be provided as a single
   file, which will be compiled as a dynamic library.
3) Each c-function is compiled as a standalone executable, that is executed from
   within the corresponding cmex-function, using the system() command.
   Data is passed between the cmex-function and the c-function by writing to a file.
4) All c-functions are compiled to a standalone executable, which acts a server
   that calls the c-functions at request of the client cmex-functions.
   Data is passed between clients and server through a socket, which permits
   clients and server to run on different computers.
 
Typically, templates are of the form:
 
\#ifdef createGateway
MEXfunction MEXmtimes
Cfunction   Cmtimes
method      mtimes
include     Cmtimes.c
 
inputs
	double X1 [m,k]
	uint32 S1 [1]
	double X2 [k,n]
	uint32 S2 [1]
 
outputs
	double Y [m,n]
 
preprocess(V1,V2)
 \{ ... Matlab code ...\}
 
MEXfunction MEXplus
Cfunction   Cplus
method      plus
include     Cplus.c
 
inputs
	double X1 [m,n]
	uint32 S1 [1]
	double X2 [m,n]
	uint32 S2 [1]
 
outputs
	double Y [m,n]
 
preprocess(V1,V2)
 \{ ... Matlab code ...\}
 
\#endif
 
void Cmtimes(double \textbf{X1,uint32\_t }S1,double \textbf{X2,uint32\_t }S2,double *Y,
              mwSize m,mwSize k,mwSize n)
\{ ... C code ... \}
 
void Cplus(double \textbf{X1,uint32\_t }S1,double \textbf{X2,uint32\_t }S2,double *Y,
           mwSize mmwSize n)
\{ ... C code ... \}
 
The 'MEXfunction' statement defines the name of the cmex-function
to be created. The same template may define several cmex-functions,
each definition starts with a 'MEXfunction' statement.
 
The 'inputs' section defines inputs to the cmex-function. Pointers to
the storage space of these variables will be available to the c-function.
 
The 'outputs' section defines the outputs of the cmex-function,
these variable will be created and pointers to the storage space of
these variables will be available to the c-function.
  
Each entry of the 'inputs' and 'outputs' section is of the form
   \{matlab type\} \{variable name\} [ \{array size\} ]
 
The \{array size\} may contain specific numerical values, or
size-variable names. When the same size-variable appears multiple
times, the C gateway function will check for consistency.
The size-variable will be available with the C gateway with type 'mwSize'.
In the 'inputs' section, the \{array size\} may contain the symbol '~'
for some of the dimensions, which means that the dimension will not be
checked by the gateway function.
In the 'outputs' section, the \{array size\} may contain valid C
expressions involving size-variables.
If the \{array size\} of the outputs section is equal to '~',
then the C gateway function will not allocate storage space,
which will have to be done by the C function.
 
The (optional) 'preprocess' section defines matlab code that
will be executed after the cmex function is created, but before
the gateway function is written to the file. The code will be evaluated
as a function with the given arguments as input parameters.
This matlab code may be used to generate C code. When this code
is appended to the file 'cmexfunction', it will appear after
the 'defines', but before the gateway function.
 
The (optional) 'include' section defines functions that
should be included (via \#include directives) just before the gateway function.
more than one 'include' section are possible, resulting in multiple
includes.
 
The 'Cfunction' statement defines the name of the
C function that carries out the computations. This function will
be called from inside the gateway function with arguments determined
from the 'inputs' section, including:
  1) pointers to the input variables (data portion)
  2) pointers to the output variables (data portion)
     (automatically allocated)
  3) variables with array sizes
 
The (optional) method statement defines the name of the method
that calls the cmex function, when the parameter 'className' is non-empty,
in which case a matlab class is created to call all the cmex-functions.
 
The initial \#ifdef and the final \#endif are optional. When present,
only the portion of the file between these commands is processed,
otherwise the whole file is processed. 

\paragraph{Input parameters:}

\begin{itemize}
\item \texttt{verboseLevel} [default \texttt{0}]

   Level of verbose for debug outputs (\texttt{0} - for no debug output)

\item \texttt{parametersStructure} [default \texttt{''}]

   Structure whose fields are used to initialize parameters   
not present in the list of parameters passed to the function.   
This structure should contains fields with names that match   
the name of the parameters to be initialized.

\item \texttt{pedigreeClass} [default \texttt{''}]

   When nonempty, the function outputs are saved to a file set.   
All files in the set will be characterized by a 'pedigree',   
which decribes all the input parameters that were used in the script.   
This variable contains the name of the file class and may include a path.   
See also createPedigree

\item \texttt{executeScript} [default \texttt{'yes'}] taking values in [\texttt{'yes'},\texttt{'no'},\texttt{'asneeded'}]

   Determines whether or not the body of the function should be executed:   
\begin{itemize}
      
\item \texttt{yes} - the function body should always be executed.      
\item \texttt{no}  - the function body should never be executed and therefore the      
          function returns after processing all the input parameters.      
\item \texttt{asneeded} - if a pedigree file exists that match all the input parameters      
               (as well as all the parameters of all 'upstream' functions)      
               the function body is not executed, otherwise it is execute.\end{itemize}


\item \texttt{template}

   Filename with the template function.   
    
Alternatively, templates can be given as a structure array of the form:   
 template=struct(...   
            'MEXfunction',\{\},... \% string = name of the cmex function to be created   
            'Cfunction',\{\},...   \% string = name of the C function that carries out the computation   
            'method',\{\},...      \% string = name of the matlab method that call the cmex function   
                                   \%          only used when 'className' is non-empty   
            'inputs',struct(...      
                'type',\{\},...    \% string   
                'name',\{\},...    \% cell-array of strings (one per dimension)   
                'sizes',\{\}),...  \% cell-array of strings/or numeric array (one per dimension)   
            'outputs',struct(...  \% string   
                'type',\{\},...    \% string   
                'name',\{\},...    \% cell-array of strings (one per dimension)   
                'sizes',\{\}),...  \% cell-array of strings/or numeric array (one per dimension)   
            'preprocess',\{\}..    \% strings (starting with parameters in parenthesis)   
            'includes',\{\}..      \% cell-array of strings (one per file)   
            );   
in which the different fields of the structure map directly   
to the corresponding sections of the template file.

\item \texttt{preprocessParameters} [default \texttt{\{\}}]

   Cell array containing the input parameters to the function(s) defined   
in the 'preprocess' section(s).

\item \texttt{defines} [default \texttt{\{\}}]

   Matlab structure that specifies a set of \#define   
preprocessor directives that will be included before the   
the \#include directive and also before the gateway function   
These directives can be used to pass (hardcoded) parameters.   
Should be of the form:   
      defines.name1 = \{string or scalar\}   
      defines.name2 = \{string or scalar\}   
     .... 

\item \texttt{callType} [default \texttt{'include'}] taking values in [\texttt{'include'},\texttt{'dynamicLibrary'},\texttt{'standalone'},\texttt{'client-server'}]

   Determines how the gateway function should call the C function that   
actually performs the computations. It can take the following values   
'include' - The gateway function should call 'Cfunction',   
              which must have been defined in the 'includes'   
              that precede the declaration of the gateway function.   
              The C function will thus be statically linked to the   
              gateway function.   
'dynamicLibrary' - Each time the gateway function is called, it:   
              (1) The first time a gateway function is called it   
                  loads the dynamic library defined by 'dynamicLibrary'   
                  (this step is not needed in subsequent calls).   
              (2) calls 'Cfunction' that must exist within the library.   
              (3) does NOT unload the dynamic library   
              A mexFunction   
                 function rc=dynamicLibrary\_load(boolean)   
              is automatically created to load or unload the library   
              It should be called (with boolean=false)   
              when the gateways are no longer needed.   
              When the cmex functions are incorporated into a matlab class   
              (see 'className' parameters), the creating of the matlab   
              object automatically calls dynamicLibrary\_load(true);   
              BUT the destruction of the object does not call   
              dynamicLibrary\_load(false), which should be done manually.;   
              ATTENTION: the gateway will CRASH matlab if called after   
              the library is unloaded.   
'standalone' - The gateway function performs the following tasks:   
              (1) writes all input variables to a file,   
              (2) calls a standalone executable that reads the inputs   
                  and writes the outputs to a file   
              (3) reads the outputs from the file created by the   
                  standalone executable and returns them   
              The standalone executable is automatically created   
              and calls the 'Cfunction,' which are expected to be defined   
              in the 'includes'.   
              The file IO introduces large overhead, but it is useful for   
              debugging and code profiling.   
'client-server' - The gateway function performs the following tasks:   
              (1) opens an I/O pipe to communicate with a server that   
                  performs the computations   
              (2) writes all input variables to the pipe,   
              (3) reads the outputs from the pipe and returns them   
              The server executable is automatically created   
              based on 'CfunctionsSource,' which is expected to be defined   
              When 'className' is defined, a method 'upload' is   
              automatically created to upload/compile/execute the server   
              in 'serverAddress' using ssh.   
              ATTENTION: Currently the serverIP and port are hardcoded   
                         in the mex files.   
ATTENTION: for both dynamicLibrary types, the types for the parameters   
           of 'Cfunction' are not checked by the compiler.

\item \texttt{CfunctionsSource} [default \texttt{''}]

   Name of a .c source file that contains the code for all the c-functions.   
This parameter is used only when 'callType' has one of the values:   
'dynamicLibrary'      - source code for the dynamic library   
'client-server'       - source code that will be included in the    
                          server executable.

\item \texttt{folder} [default \texttt{'.'}]

   Path to the folder where the files will be created.   
Needs to be in the Matlab path.

\item \texttt{dynamicLibrary} [default \texttt{''}]

   Name of the dynamic library (without extension) to be created.   
This parameter is used only when 'callType'='dynamicLibrary'.

\item \texttt{serverProgramName} [default \texttt{''}]

   Name of the executable file for the server executable.   
This parameter is used only when 'callType'='client-server'.

\item \texttt{serverAddress} [default \texttt{'localhost'}]

   IP address (or name) of the server.   
This parameter is used only when 'callType'='client-server'.

\item \texttt{port} [default \texttt{1968}]

   Port number for the socket that connects client and server.   
This parameter is used only when 'callType'='client-server'.

\item \texttt{compileGateways} [default \texttt{true}] taking values in [\texttt{true},\texttt{false}]

   When 'true' the gateway functions are compiled using 'cmex'.

\item \texttt{compileLibrary} [default \texttt{true}] taking values in [\texttt{true},\texttt{false}]

   When 'true' the dynamicLibrary is compiled using 'gcc'.   
This parameter is used only when 'callType'='dynamicLibrary'.

\item \texttt{compileStandalones} [default \texttt{true}] taking values in [\texttt{true},\texttt{false}]

   When 'true' the standalone/server executable is compiled using 'gcc'.   
This parameter is used only when 'callType' has one of the values:   
   'standalone' or 'client-server'

\item \texttt{compilerOptimization} [default \texttt{'-Ofast'}] taking values in [\texttt{'-O0'},\texttt{'-O1'},\texttt{'-O2'},\texttt{'-O3'},\texttt{'-Ofast'}]

   Optimization flag used for compilation.   
Only used when compileGateways, compileLibrary, or compileStandalones

\item \texttt{targetComputer} [default \texttt{'maci64'}] taking values in [\texttt{'maci64'},\texttt{'glnxa64'},\texttt{'pcwin64'}]

   OS where the mex files will be compiled.

\item \texttt{serverComputer} [default \texttt{'maci64'}] taking values in [\texttt{'maci64'},\texttt{'glnxa64'},\texttt{'pcwin64'}]

   OS where the server will be compiled.   
This parameter is used only when 'callType'='client-server'.

\item \texttt{className} [default \texttt{''}]

   When non empty, a class is created that encapsulates all the   
gateway functions as methods.   
When a dynamicLibrary is used, the class creation loads the library   
and the class delete unloads the library.

\item \texttt{classHelp} [default \texttt{''}]

   When 'className' is non empty, this string or string array are   
included in the classdef file to provide an help message.

\end{itemize}

\paragraph{Outputs:}

\begin{itemize}
\item \texttt{statistics}

   Structure with various statistics, include the file sizes and compilations times
\end{itemize}


